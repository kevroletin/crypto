\documentclass[10pt,a4paper]{article}
\usepackage[cm]{fullpage}
\usepackage[utf8x]{inputenc}
\usepackage{ucs}
\usepackage{amsmath}
\usepackage{amsfonts}
\usepackage{amssymb}
\usepackage[russian]{babel}
\usepackage{listings}
\author{Кевролетин В.В.}
\title{Криптография}
\begin{document}

\maketitle

\subsection*{Задание5.3}
\subsubsection*{Условие}
Доказать свойство дополнительности DES (1): если C=DES(M,K), то
C'=DES(M',K') (Z' - обозначает слово, составленное из дополнений
соответствующих битов бинарного слова Z). (Используйте следующее
равенство для логических переменных (x+y)'=x'+y.) На первый взгляд,
для анализа DES с помощью простого перебора ключей необходимо
исследовать $$ 2^{56}  $$ вариантов. Как меняет приведенный результат эту
оценку?
\subsubsection*{Решение}

Сразу отмечу, что свойство комплиментарности приводит к тому, что для
полного перебора ключей необходимо исследовать в 2 раза меньше ключей.

Первая и последняя перестановки лишь переставляют биты, не меняя их
значения, поэтому не влияют на свойство дополнительности.

Так что надо более подробно рассмотреть механизмы генерации ключа и
прменения функции f.

Для генерации ключей раундов используются операция циклического сдвига
и перестановки. Циклический сдвиг является перестановкой и обладает
свойством дополнительности.

Итак, возьмем форулы для сети Фейстеля:
$$ L_1 = R_0 $$ 
$$ R_1 = L_0 ⊕ f(R_0 , K_0 ) $$

Если возьмём дополнения, 
$$ L_i' = \neg L_i ; R_i' = \neg R_i $$
то будем иметь:
$$ L_1' = R_0' = \neg R_0 = \neg L_0 $$
$$ R_1 = L_0 ⊕ f (R_0 , K_0 ) = \neg L0 ⊕ f (\neg R_0 ,\neg K_0) $$
Последнее равенство верно, так как $ f(R_i,K_i) $ применяет несколько
перестановок к $ R_i и K_i $, а так же выбор значения из
s-блоков: 1)как было ранее отмечено,
перестановки не меняют значения битов, а лишь переставляют их 2) перед
выборкой из s-блоков к ключу раунда и данным(прошедним расширяющую
перестановку) применятся операция  ⊕, но 
$ \neg R_i ⊕ \neg K_i = R_i ⊕ K_i $ т.е.
значение этой операции будет
тем же самым, когда мы возьмём дополнения к ключу и входным данным.
Суперпозиция нескольких операций, обладающих свойством
дополнительности опять будет обладать этим свойством, так что функция
f, а значит и весь процесс шифровиня обладает свойством
дополнительности.



\subsection*{Задание5.4}
\subsubsection*{Условие}
Пусть $$ \phi_q $$ -- подстановка, которая реализуется цикловой функцией
шифра Файстеля, $$ T^n $$-- циклический сдвиг вправо, 2n -- длина
блока. Доказать, что $$ T^n, T^n \phi_q, \phi_q T^n $$-- инволюции
\subsubsection*{Решение}
Преобразование P называется инволюцией, если для любого блока
$$ PPw = w $$

Для простоты записи договоримся, что блок w длины 2n состоят из 2х
блоков длины n: w = (A,B)

Для $ T^n $: 

$$ T^n(T^n(A,B)) = T^n(B,A) = (A,B) $$ ч.т.д.

Для $ T^n \phi_q $:

$$ T^n \phi_q((A,B)) = T^n(B, A ⊕ F(q, B)) = (A ⊕ F(q, B), B) = $$
$$ T^n \phi_q T^n \phi_q((A,B)) = T^n \phi_q (A ⊕ F(q, B), B) = $$
$$ T^n \phi_q (A ⊕ F(q, B), B) = (A ⊕ F(q, B) ⊕ F(q, B), B) = (A,B) $$ ч.т.д.

Для $ \phi_q T^n $:
$$ \phi_q(T^n(A,B)) = \phi_q(B, A) = (A ⊕ F(q, B), B) $$
Первое равенство аналогично первому равенству из предыдущего примера,
а остальное аналогично. ч.т.д.

\end{document}
