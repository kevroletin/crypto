\documentclass[10pt,a4paper]{article}
\usepackage[utf8x]{inputenc}
\usepackage{ucs}
\usepackage{amsmath}
\usepackage{amsfonts}
\usepackage{amssymb}
\usepackage[russian]{babel}
\usepackage{listings}
\author{Кевролетин В.В.}
\title{Докомпьютерные шифры}
\begin{document}

\maketitle

\subsection*{Задание3.3.1}
\subsubsection*{Условие}
Какие из докомпьютерных шифров являются групповыми, а какие нет (с
доказательством)?
\subsubsection*{Решение}
\subsubsection*{Модулярный шифр}

Дважды применим шифр к исходному символу m:\\
$$ c_1 = n_1 * m + k_1 (mod M) $$
$$ c_2 = n_2 * c_1 + k_2 (mod M) = $$
$$ = n_2 * (n_1 * m + k_1) + k_2 (mod M) = $$
$$ = n_2 * n_1 * m + n_2 * k_1 + k_2 (mod M) = $$
$$ = N * m + K (mod M), N = n_2 * n_1, K = n_2 * k_1 + k_2 $$
Таким образом шифр является групповым.

\subsubsection*{шифр Вижнера}
Применим процедуру шифрования дважды, с ключами $ K_1, K_2 $. Если
ключи одинаковой длины, то все сводится к предыдущему случаю. Если
ключи разной длины, то длина нового ключа будет равна наименьшему
общему кратному длин ключей $ K_1, K_2 $, а сам ключ выражается
несложной формулой:
$$ K_1 = k_1^1,k_2^1,...,k_{s1}^1 $$
$$ K_2 = k_1^2,k_2^2,...,k_{s1}^2 $$
Новый ключ(т.к. мы 2 раза делаем сдвиг):
$$ (K)_i = (k_{i mod s1}^1 + k_{i mod s2}^2) (mod M) $$

Т.е. шифр является групповым

\subsubsection*{шифр Плейфейера}

Приведём контрпример, показывающий, что шифр негрупповой.

Пусть первое шифрование с ключём $K_1$ переводит пару символов 
$$ AB \rightarrow CD $$
А второе шифрование с ключём $K_2$, обратно
$$ CD \rightarrow AB $$
Тогда если бы шифр был групповым, то должен был бы существовать ключ,
шифрование которым эквивалентно шифрованием с ключем $K_1$, а затем 
$K_2$. Но тогда бы пара символов AB переводилась бы сама в себя, что
невозможно в системе Плейфейера.

Таким образом, шифр не групповой.

\subsubsection*{шифр Хилла}

В этом шифре применяются линейные преобразования:

$$ \vec c_1 = A_1 * \vec m + \vec b_1 (mod M) $$
$$ \vec c_2 = A_2 * \vec c_1 + \vec b_2 (mod M) $$
$$ \vec c_2 = A_2 * (A_1 * \vec m + \vec b_1) + \vec b_2 = $$
$$ A_1*A_2 \vec m + A_2 * \vec b_1 + \vec b_2 = $$
$$ A * \vec m + \vec b, A = A_1*A_2, \vec b = A_2 * \vec b_1 + \vec b_2 $$

Таким образом, шифр групповой.


\subsection*{Задание3.3.2}
\subsubsection*{Условие}
Показать, что шифр перестановки является линейным преобразованием в
$ B^n, B={0,1} $
\subsubsection*{Решение}
Перестановку n-элементного вектора можно представить матрицей
размерности n*n, где в каждой строке один элемент, равной 1, а
остальные нули, и в каждом столбце один элемент, равной 1, а
остальные нули.

\subsection*{Задание3.3.3}
\subsubsection*{Условие}
Сколько существует нелинейных криптопреобразований $ B^3->B^3 $? 
\subsubsection*{Решение}
Посчитаем сколько всего приобразований:
существует $ 2^3 $ разных бинарных векторов длины n => существует $
2^3! $ различных преобразований $ B^3->B^3 $.
Чтобы посчитать количество линейных преобразований в $ B^3 $
необходимо посчитать число различных  бинарных матрицы размерности
3х3: $ 2^9 $. Таким образом нелинейных преобразований: $ 2^3! - 2^9 $

\subsection*{Задание3.3.4}
\subsubsection*{Условие}
Доказать, что перестановка П:$ 0, 1, 2, … , 2^n-1 $(взаимно однозначное
отображение n-битовых целых чисел в себя) в более, чем 60\% случаев
\% имеет неподвижную точку, $ n \geq 2 $
\subsubsection*{Решение}
Как было отмечено ранее, перестановку можно записать в виде матрицы,
имеющей в каждой строке и столбце одну 1, а остальные 0.
Если перестановка имеет неподвижную точку, то она имеет хотя бы одну 1
на диагонали. Таким образом задача сводится к подсчёту нужных нам
матриц с 1 на диагонали.
Для подсчёта будем использовать принцип включения и
исключения(множество матриц, содержащих хотя бы одну 1 на диагонали
включает множество матриц с хотя бы 2мя единицами на диагонали и т.д)

Количество вершин без неподвижной точки выражается формулой (которая
называется субфакториалом числа)
$$ !n = n!-\frac{n!}{1!}+\frac{n!}{2!}-\frac{n!}{3!}+\dots
+(-1)^n\frac{n!}{n!} = \sum_{k=0}^n(-1)^k\frac{n!}{k!}  $$

Известно, что субфакториал асимптотически ведёт себя, как 
$ \frac{n!}{e} $

Тогда при больших n процент перестановок без неподвижной точки
$$ \frac{\frac{n!}{e}}{n!} = \frac{1}{e} $$
Соответственно процент перестановок с неподвижной точкой примерно
$$ 1 - \frac{1}{e} \approx 0,63 $$
ч.т.д.


\end{document}
