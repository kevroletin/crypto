\documentclass[10pt,a4paper]{article}
\usepackage[utf8x]{inputenc}
\usepackage{ucs}
\usepackage{amsmath}
\usepackage{amsfonts}
\usepackage{amssymb}
\usepackage[russian]{babel}
\usepackage{listings}
\author{Кевролетин В.В.}
\title{Теория сложности вычислений }
\begin{document}

\maketitle

\subsection*{Задание6.1}
\subsubsection*{Условие}
Разработать алгоритм с полиномиальным (от размерности) временем работы
для решения проблемы выполнимости булевой функции, заданной КНФ
(конъюнкция дизъюнкций), в которой любая дизъюнкция содержит ровно два
литерала (литерал - переменная или ее отрицание). Использовать тот
факт, что если один из литералов имеет значение "ложь", то второй
должен иметь значение "истина" для того, чтобы КНФ приняла значение
"истина" на соответствующем наборе логических значений. 
\subsubsection*{Решение}
Алгоритм - перебор с возвратом:
Последовательно рассматриваем каждую дизъюнкцию. Для каждой дизъюнкции
подставим в значение "истина" для первой или второй или для обеих
переменных (только при условии, что соответствующая переменная еще не
получила значение, пока мы рассматримали предыдущие дизъюнкции). Если
обе переменные уже имеют значение "ложь", то возвращаемся на шаг
назад. Если же хотя бы одна из переменных "истина" - продолжаем
подставлять значения в следующие дихъюнкциях. \\
Пусть выражение состоит из n дизъюнкций, соединенных
конъюнкциями. Тогда мы получаем экспоненциальный алгоритм....


\begin{lstlisting}
\end{lstlisting}


\subsection*{Задание}
\subsubsection*{Условие}
\subsubsection*{Решение}

\subsection*{Задание}
\subsubsection*{Условие}
\subsubsection*{Решение}

\end{document}
