\documentclass[10pt,a4paper]{article}
\usepackage[cm]{fullpage}
\usepackage[utf8x]{inputenc}
\usepackage{ucs}
\usepackage{amsmath}
\usepackage{amsfonts}
\usepackage{amssymb}
\usepackage[russian]{babel}
\usepackage{listings}
\author{Кевролетин В.В.}
\title{RSA}
\begin{document}

\maketitle

\subsection*{Задание8.1}
\subsubsection*{Условие}
Доказать, что $a^{-1}$ mod m существует тогда и только тогда, когда
нод(a,m)=1.
\subsubsection*{Решение}
\subsubsection*{Необходимость}
От противного, допустим  НОД(a,m) = d > 1
$$ a*a^{-1} = 1 (mod m) $$
$$ a*a^{-1} = 1 + q*m $$
$$ a*a^{-1} - q*m = 1 $$
Левая часть делится на d > 1, тогда и правая часть должна делиться на
d, но справа стоит 1. Противоречие.
\subsubsection*{Достаточность}
Запишем линейное представление НОД:
$$ GCD(a, m) = a*x + m*y $$
$$ a*x + m*y = 1 $$
$$ a*x = 1 - m*y $$
$$ a*x = 1 (mod\ m) $$
т.е. x = $a^{-1} (mod\ m)$

\subsection*{Задание8.2}
\subsubsection*{Условие}
Кольцо классов вычетов по mod m $(Z_m)$ является полем тогда и только
тогда, когда m - простое.
\subsubsection*{Решение}
\subsubsection*{Необходимость}
Пусть поле содержит m элементов. Тогда каждый ненулевой элемент $x_i$
имеет обратный, т.е.(по результатам предыдущего упражнения) $GCD(x_i,m)
= 1$. Таким образом, функция Эйлера 
$\phi (m) = m-1$ $\Rightarrow$ m - простое
\subsubsection*{Достаточность}
Коммутативное и ассоциативное кольцо R с единицей называется полем, если
каждый ненулевой элемент a ∈ R обладает обратным, то есть существует
такой элемент $a^{−1}$ , что $a^{−1} a = aa^{−1} = 1$.

Выполнение этого условия вытекает из результата предущего упражнения -
НОД($x_i$,m) = 1 $\Rightarrow$ существует обратный элемент.
НОД($x_i$,m) = 1 для любого ненулевого $x_i$ по определению т.к. m - простое.

\subsection*{Задание8.3}
\subsubsection*{Условие}
Предположим, что найден эффективный способ решения задачи нахождения d
по e. Означает ли это, что можно решать эффективно задачу факторизации
(нахождения p и q по n).
\subsubsection*{Решение}
$$ Ed = 1 (mod\ (p - 1)(q - 1)) $$
$$ Ed - 1 = s(p - 1)(q - 1) $$
Возьмём произвольное целове число $ X \neq 0 $, тогда по малой теореме Ферма:
$$ X^{Ed-1} = 1 (mod\ N) $$
$ Ed - 1 $ (т.к. (p - 1)(q - 1) четно) значит можем взять квадратный корень:
$$ Y_1 = X^{(Ed-1)/2} = 1 (mod\ N) $$
$$ Y_1^2 = 1 (mod\ N) $$
$$ Y_1^2 - 1 = k*N $$
$$ (Y_1 - 1)(Y_1 + 1) = k*N $$
Посчитаем НОД($Y_1 - 1$, N), НОД($Y_1 + 1$, N) - если получили число
отлично от 1 - задача решена. Если же не повезло возьмём квадратный 
корень еще раз $ Y_2 = X^{(Ed-1)/4} $. И повторим процедуру. Если не
повезло второй раз - выберем другое число X

\subsection*{Задание8.4}
\subsubsection*{Условие}
Показать, что заданный алгоритм осуществляет возведение в степень с
использованием метода последовательного возведения в квадрат.
\subsubsection*{Решение}
Рассмотрим 2-ичное разложение числа n:
$$    n=m_{k} \cdot 2^{k}+m_{k-1} \cdot 2^{k-1}+\dots+m_{1} \cdot 2+m_{0} $$
Подставим в  $ x^n $:
$$  x^{n}=x^{((\dots((m_{k} \cdot 2+m_{k-1}) \cdot 2+m_{k-2}) \cdot
  2+\dots) \cdot 2+m_{1}) \cdot 2 + m_{0}}=((\dots(((x^{m_{k}})^{2}
\cdot x^{m_{k-1}})^{2}\dots)^{2} \cdot x^{m_{1}})^2 \cdot x^{m_{0}} $$
Основываясь на полученном выражении, можно последовательно возводить в
степень:
$$(1) x^{m_k*2} $$
$$(2) x^{m_k*2 + m_{k-1}} $$
$$(3) x^{(m_k*2 + m_{k-1})*2} $$
$$(4) x^{(m_k*2 + m_{k-1})*2 + m_{k-2}} $$
$$...$$
Запишем этот алгоритм в виде процедуры на языке программирования Перл:
\begin{lstlisting}
1   sub fast_pow {
2      my ($a, $b, $m) = @_;
3      my $x = 1;
4      my $i = length(sprintf("%b", $b));
5      while (--$i >= 0 ) {
6          $x = ($x * $x);
7          $x = ($x * $a) if ($b >> $i) & 1;
8      }
9      $x
10  }
\end{lstlisting}
Строки 4,5 задают цикл от самого значимого бита числа n до менее
значимого. В цикле выбирается соответствующая цифра двоичного разложения
числа n и производится последовательное возведение в степень. Шагам
1,3 приведённого выше примера соответствует 6я строка в коде. Шагам
2,4 - 7я строка.

\subsection*{Задание8.5}
\subsubsection*{Условие}
Исполнить WITNESS при a=7, p=561.
\subsubsection*{Решение}
Яп реализации - Перл:
\begin{lstlisting}
use bigint;
use warnings;
use strict;

sub witness {
    my ($n, $k) = @_;
    return 1 if $n == 2;
    return 0 if $n < 2 || $n % 2 == 0;

    my ($d, $s) = ($n - 1, 0);

    while (!($d % 2)) {
        $d /= 2;
        $s++;
    }

 LOOP: for(1 .. $k) {
        my $a = 2 + int(rand($n-2));

        my $x = $a->bmodpow($d, $n);
        next if $x == 1 || $x == $n-1;

        for (1 .. $s-1) {
            $x = ($x*$x) % $n;
            return 0 if $x == 1;
            next LOOP if $x == $n-1;
        }
        return 0;
    }
    1
}

print witness(561, 7);
\end{lstlisting}
Результат - 0, т.е. тест показал, что число 561 составное (561 = 3*11*17)к

\subsection*{Задание8.6}
\subsubsection*{Условие}
Найти количество составных натуральных чисел a, не превосходящих 561
таких, что $a^{560}=1 mod 561$.
\subsubsection*{Решение}
561 = 3*11*17 $\Rightarrow$ вместо проверки равенства $a^{560} = 1
(mod 561)$ можно проверить, выполняются ли одновременно
\begin{eqnarray*}
a^{560} & = & 1 (mod\ 3)  \\
a^{560} & = & 1 (mod\ 11)  \\
a^{560} & = & 1 (mod\ 17)  
\end{eqnarray*}
Малая теорема Ферма:
Если p — простое число, и целое a не делится на p, то 
$a^{p-1} ≡ 1 (mod\ p)$, т.е.
\begin{eqnarray*}
a^{2} & = & 1 (mod\ 3)  \\
a^{10} & = & 1 (mod\ 11)  \\
a^{16} & = & 1 (mod\ 17)  
\end{eqnarray*}
Тогда, т.к. 2|560, 10|560, 16|560, получается, что первая система
равенств выполняется для всех чисел, не кратных 3,11 или 17. Перебором
получим результат: 320
\begin{lstlisting}
my $a=0; 
for (1..560) { ++$a if $_%3 && $_%1 && $_%17 } 
print $a
\end{lstlisting}


\subsection*{Задание8.7}
\subsubsection*{Условие}
Инвариант цикла в EXTENDED EUCLID.
\subsubsection*{Решение}
На каждой итерации цикла x и y получают значения такие, что
$$ a \cdot x + b \cdot y = g $$
Мы нашли решение $(x_1,y_1)$ задачи для пары $(b\%a,a)$, такое что
$$ (b \% a) \cdot x_1 + a \cdot y_1 = g, $$
на предыдущей итерации цикла.
Покажем, что решение (x,y) для нашей пары (a,b) вычисляются корректно:
$$ b \% a = b - \left\lfloor \frac{b}{a} \right\rfloor \cdot a $$
Подставим это в приведённое выше выражение с $x_1$ и $y_1$ и получим:
$$ g = (b \% a) \cdot x_1 + a \cdot y_1 = $$
\[
\left( b - \left\lfloor \frac{b}{a} \right\rfloor \cdot a \right)
 x_1 + a \cdot y_1 
\]

$$ g = b \cdot x_1 + a \cdot \left\lfloor \frac{b}{a} \right\rfloor
\cdot x_1   $$
Сравнивая это с исходным выражением над неизвестными x и y, получаем требуемые выражения:

\[
\begin{cases}
x = y_1 - \left\lfloor \frac{b}{a} \right\rfloor x_1 \\
x = x_1
\end{cases}
\]

\subsection*{Задание8.8}
\subsubsection*{Условие}
Найти нод(560,1769) с использованием расширенного алгоритма
Евклида.
\subsubsection*{Решение}
Ответ: 1 = 477*560 + -151*1769 \\
Код:
\begin{lstlisting}
sub ext_gcd {
    my ($a, $b) = @_;

    if ($a == 0) {
        return ($b, 0, 1)
    }
    my ($d, $x1, $y1) = ext_gcd($b % $a, $a);
    ($d, $y1 - int($b / $a) * $x1, $x1)
}


my ($a, $b) = (560, 1769);
printf "%d = %d*$a + %d*$b", ext_gcd($a, $b);
\end{lstlisting}

\subsection*{Задание8.9}
\subsubsection*{Условие}
Доказать, что если n - простое (>2), то n делит $2^n-2$. Доказать, что
составное число 341 делит $2^{341}-2$.
\subsubsection*{Решение}
$$ 2^n - 2 = x (mod\ n) $$
$$ 2^n = 2 + x (mod\ n) $$
$$ 2^{n-1} = \frac{2 + x}{2} (mod\ n) $$
Основываясь, на малой теореме Ферма(т.к. n - простое):
$$ \frac{2 + x}{2} = 1 (mod\ n) $$
$$ x/2 = 0 (mod\ n) $$
$$ x = 0 (mod\ n) $$
ч.т.д

В уравнении $ 2^{341} - 2 = x (mod\ 341) $ найдём x. Т.к. 341 = 11*31,
то вместо $ 2^{341} = x + 2 (mod\ 341) $ мы можем решать
\begin{eqnarray*}
2^{341} & = & x + 2 (mod\ 11)  \\
2^{341} & = & x + 2 (mod\ 31)  \\
\end{eqnarray*}
По малой теорема Ферма:
$$ 2^{10} = 1 (mod\ 11)\ | * 43 $$
$$ 2^{340} = 1 (mod\ 11)\ | * 2$$
$$ 2^{341} = 2 (mod\ 11)\ $$
$$ 2^{341} = 0 + 2 (mod\ 11)\ $$
Для 2го уравнения:
$$ 2^{30} = 1 (mod\ 31)\ | * 11 $$
$$ 2^{330} = 1 (mod\ 31)\ | * 2$$
т.к. $ 2^{11} = 2048 = 2 (mod\ 31) $
$$ 2^{341} = 2 (mod\ 31)\ $$
$$ 2^{341} = 0 + 2 (mod\ 31)\ $$
Таким образом x = 0, т.е.
$ 2^{341} - 2 = 0 (mod\ 341) $
ч.т.д.

\subsection*{Задание8.10}
\subsubsection*{Условие}
Уравнение ax=b mod m, нод(a,m)=d>1, имеет решение тогда и только
тогда, когда d|b. Если условие выполняется, то имеется ровно d решений
по mod m.
\subsubsection*{Решение} 
\subsubsection*{Необходимость}
$$ a*x = b (mod\ m) $$
$$ a*x = b + q*m $$
$$ a*x - q*m = b $$
Левая часть делится на d, правая так же должна делиться на d.
\subsubsection*{Достаточность}
$$ a*x - q*m = b | /d $$
$$ a'*x - q'*m = b' | /d $$
$$ a'*x = b' (mod\ m') $$
т.к. НОД(a', m') = 1 $\Rightarrow$ существует $ a'^{-1}\ mod\ m$
$$ x = b'*a'^{-1} (mod\ m') $$
Покажем, что таких решений будет d штук.
$ m = m'd \Rightarrow x = a'^{-1}b' + m'*q, q = 0,...,d-1 $ - разные
значения по mod m

\subsection*{Задание8.11}
\subsubsection*{Условие}
Решить систему x=2 mod 3, x=3 mod 5, x= 2 mod 7.
\subsubsection*{Решение}
x = 23

\subsection*{Задание8.12}
\subsubsection*{Условие}
Шесть профессоров начинают читать лекции по своим курсам в ПН, ВТ, СР,
ЧТ, ПТ, СБ и читают их далее через 2, 3, 4, 1, 6, 5 дней
соответственно. Лекции не читаются по ВС (отменяются). Когда в первый
раз все лекции выпадут на ВС и будут отменены.
\subsubsection*{Решение}
Пусть x - количество прошедших дней с первого воскресенья

\[
\left\{ \begin{array}{llll}
x& = &0 &(mod\ 7)\\
x - 1& = &0 & (mod\ 2)\\
x - 2& = &0 &(mod\ 3)\\
x - 3& = &0 &(mod\ 4)\\
x - 4& = &0 &(mod\ 1)\\
x - 5& = &0 &(mod\ 6)\\
x - 6& = &0 &(mod\ 5)\\
\end{array}
\right. \\
\]

5-е сравнение можно выбросить, т.к. x $\neq 0$

\[
\left\{ \begin{array}{llll}
x& = &0 &(mod\ 7)\\
x & = &1& (mod\ 2)\\
x & = &2& (mod\ 3)\\
x & = &3& (mod\ 4)\\
x & = &5& (mod\ 6)\\
x & = &1& (mod\ 5)\\
\end{array}
\right. \\
\]
2е сравнение избыточно, т.к. оно включено в 4е. Аналогично 3е учтено в
6м. 

\[
\left\{ \begin{array}{llll}
x& = &0 &(mod\ 7)\\
x & = &3 &(mod\ 4)\\
x & = &5 &(mod\ 6)\\
x & = &1 &(mod\ 5)\\
\end{array}
\right. \\
\]

Рассмотрим 2е и 3е сравнения:
\[
\left\{ \begin{array}{llll}
x & = &3 &(mod\ 4)\\
x & = &5 &(mod\ 6)\\
\end{array}
\right. \\
\]
\[
\left\{ \begin{array}{llll}
x & = &-1 &(mod\ 4)\\
x & = &-1 &(mod\ 6)\\
\end{array}
\right. \\
\]
Тогда их можно заменить одним:
\[
\left\{ \begin{array}{llll}
x & = &-1 &(mod\ 12)\\
\end{array}
\right. \\
\]

В итоге имеем:
\[
\left\{ \begin{array}{llll}
x& = &0 &(mod\ 7)\\
x & = &11 &(mod\ 12)\\
x & = &1 &(mod\ 5)\\
\end{array}
\right. \\
\]

Решим полученнуюл систему сравнений, используя греко-китайскую
теорему:
$$ M_1 = 5 * 7 = 35, M_1^{-1} (mod\ 12) = 11\ (mod\ 12) $$
$$ M_2 = 12 * 7 = 84, M_2^{-1} (mod\ 5) = 4\ (mod\ 5) $$
$$ M_3 = 12 * 5 = 60, M_3^{-1} (mod\ 7) = 2\ (mod\ 7) $$
$$ M = 12 *5 * 7 = 420 $$
$$ x = 35*11*11 + 84*4*1 + 60*2*0 (mod\ 420) = 371 (mod\ 420) $$
Т.е. через 371 день после первого воскресенья


\subsection*{Задание8.13}
\subsubsection*{Условие}
Найти (678*973)mod 1813 (с использованием греко-китайской теоремы).
\subsubsection*{Решение}
$$ 1813 = 7^2 * 37, 678 = 2*3*113, 973 = 7*139 $$
$$ 678*973 (mod\ 1813)  = 2*3*113*7*139 (mod\ 7^2*37) $$
$$ = 94242 (mod\ 7^2*37) $$

Вычислим:
\[
\begin{array}{llll}
94242 & = & 1 &(mod\ 7)\\
94242 & = &11 &(mod\ 37)\\
\end{array}
\]
$ 37^{-1} = 4 (mod\ 7) $ \\
Используем формулу для решения:
$$ x = (m_2^{-1} mod\ m_1)(a_1 - a_2)m_2 + a_2 mod(m_1 * m_2) $$
$$ x = 4*(1 - 3)*37 + 3 =  -293 = 225 (mod\ 259) $$
$$ 7*225 = 1575 (mod\ 1813) $$

Ответ:\\
$$ 678*973 = 1575 (mod\ 1813) $$

\subsection*{Задание8.14}
\subsubsection*{Условие}
Вычислить первые 20 простых чисел Мерсенна.

\subsection*{Задание8.15}
\subsubsection*{Условие}
Как повлияет на работу RSA тот факт, что одно из чисел (например, p)
не является простым, а представляется в виде произведения двух
простых: $p=p_1*p_2$. 
\subsubsection*{Решение}
Используя каноническое разложение 
$ n = \prod_{i=1}^k p_i^{\alpha_i} $ числа n,
функция Эйлера может быть вычислена по формуле
$$  \varphi(n) = \prod_{i=1}^k p_i^{\alpha_i - 1} \left( p_i - 1 \right) $$

Таким образом, мы можем посчитать функцию Эйлера, если в каноническом
разложении n присутствуют 3 числа. В алгоритме, кроме как в вычислении
функции Эйлера делители числа n участия не принимают, так что всё
остаётся, как и прежде. Единственно что изменится - чем больше
делителей имеет число n, тем, потенциально, проще его факторизовать,
значит криптостойкость снижается.

\subsection*{Задание8.16}
\subsubsection*{Условие}
Как повлияет на работу RSA тот факт, что шифруемое число не является
взаимно простым, например, с p.
\subsubsection*{Решение}
Возможно 2 варианта: \\
1)p не является простым число. См. предыдущее
упрожнение.\\
2)Шифруемое число кратно p. Обычная ситуация.

Если же криптоаналитик точно знает, что шифруемое число имеет общий
делитель с p, то он применит алгоритм Эвклида и в случае \#2 получит
само число p, а в 1м случае получит одни из сомножителей p, что
качественно не упростит задача факторизации N.

\subsection*{Задание8.17}
\subsubsection*{Условие}
Сгенерировать RSA и провести шифрование/дешифрование (Mathematica,
Scheme, Sage). 

\subsection*{Задание8.18}
\subsubsection*{Условие}
Пусть n(=pq) и $\phi(n)$ известны, а p и q -- неизвестны. Выразить p и
q через n и $\phi(n)$. Рассмотреть случай n=2993 и $\phi(n)=2880$.

\subsection*{Задание8.19}
\subsubsection*{Условие}
$p,q,e,d,n$ -- параметры RSA.  Доказать, что имеется $r+s+rs$
неподвижных точек $x$, $1\leq x\leq n-1$, где $r=gcd(p-1,e-1),
s=gcd(q-1,e-1)$. (Из-за этого выбираются $p$ и $q$, для которых $r$ и
$s$ малы.) 


\end{document}
