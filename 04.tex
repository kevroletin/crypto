\documentclass[10pt,a4paper]{article}
\usepackage[utf8x]{inputenc}
\usepackage{ucs}
\usepackage{amsmath}
\usepackage{amsfonts}
\usepackage{amssymb}
\usepackage[russian]{babel}
\usepackage{listings}
\author{Кевролетин В.В.}
\title{Шифр Хилла}
\begin{document}

\maketitle

\subsection*{Задание 3.2.1}
\subsubsection*{Условие}
Каковы необходимые и достаточные условия взаимной однозначности
преобразования Хилла?
\subsubsection*{Решение}
Существование обратной матрицы для матрицы при помощи котороый
осуществляется шифрование.

\subsection*{Задание 3.2.2}
\subsubsection*{Условие}
Если det A= -1 mod 26, то A -- 2*2 матрица инволюций т. и т. т., когда
$a_{11} + a_{22}=0 mod 26$. Построить матрицу инволюций при $a_{11}=2$
\subsubsection*{Решение}
$$A = \begin{pmatrix}
2 & 1 \\
1 & 13 \\        
\end{pmatrix}$$
Проверяем:
$$ det(A) (mod\ 26) = 2*13 - 1 (mod\ 26) = -1(mod\ 26)$$


\subsection*{Задание 3.2.3}
\subsubsection*{Условие}
Какова методика криптоанализа шифра Хилла с избранным открытым
текстом?
\subsubsection*{Решение}
При известном открытом тексте, размерности матрицы и криптотексте
строится система линейных уравнений, решение которой даёт обратную
матрицу. Продемонстрируем на примере:

Открытый тест: HELP

Криптотекст: HIAT

Тогда, обозначив за M исходную матрицу, используемую при шифровании,
и как и при шифровании записав вместо букс их порядковые номера
получим:

$$ M \begin{pmatrix} 7 \\ 4 \end{pmatrix} = 
\begin{pmatrix} 7 \\ 8 \end{pmatrix}$$

$$ M \begin{pmatrix} 11 \\ 15 \end{pmatrix} = 
\begin{pmatrix} 0 \\ 9 \end{pmatrix}$$

Решив данную системы найдём M:

$$\begin{pmatrix} 
2 & 3 \\
2 & 5
\end{pmatrix} $$


\end{document}
