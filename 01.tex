\documentclass[10pt,a4paper]{article}
\usepackage[cm]{fullpage}
\usepackage[utf8x]{inputenc}
\usepackage{ucs}
\usepackage{amsmath}
\usepackage{amsfonts}
\usepackage{amssymb}
\usepackage[russian]{babel}
\usepackage{listings}
\author{Кевролетин В.В.}
\title{Криптография}
\begin{document}

\maketitle

\subsection*{Задание 1.1}
\subsubsection*{Условие}
Сколько возможных ключей позволяет использовать шифр Плейфейра?
(Представить в виде степени двойки.)

\subsubsection*{Решение}
Если не ограничиваться представлением ключа в виде таблицы 5х5, то
существует $ C_{25}^2 $ пар букв для алфавита из 25ти символов. Тогда
существует $ C_{25}^2! $ способов поставить в соответсвие одной паре
другую.

Если же рассматривать только способ задания ключа в виде
квадратной таблицы, то существует $ 25! $ способов заполнить
квадратную таблицу символами (для алфавита из 25ти символов). При этом
если производить циклический сдвиг строк или столбцов ключ не
изменится, поэтому чтобы избавиться от разных записей одного и того же
ключа необходимо разделить на поличество строк и столбцов. В итоге
получаем $ \frac{25!}{5*5} $ различных ключей. 

\subsection*{Задание 1.2}
\subsubsection*{Условие}
Реализовать (Mathematica, Scheme, Sage, ...) схему шифрования-дешифрования Плейфейера, подготовить тесты по методу белого ящика, продемонстрировать его работу и методику криптоанализа на достаточно длинном шифрованном тексте. 
\subsubsection*{Решение}
Язык реализации: Перл
\begin{lstlisting}

use warnings;
use strict;

sub split_to_pairs {
    my ($t) = @_;
    my @res;
    my $last = '';
    for (split //, $t) {
        if ($last) {
            push @res, [$last, $_];
            $last = undef;
        } else {
            $last = $_
        }
    }
    @res
}

sub table {
    my ($key) = @_;
    my %used = ('J' => 1);
    my ($t, $h);
    my $i = 0;
    for ((split //, $key), 'A' .. 'Z') {
        next if $used{$_};
        $t->[$i / 5][$i % 5] = $_;
        $h->{$_} = {x => $i % 5, y => int $i / 5};
        $used{$_} = 1;
        ++$i;
    }
    { table => $t, coords => $h}
}

sub encode_pair {
    my ($t, $pair) = @_;
    my ($a, $b) = (map { $t->{coords}->{$_} } @$pair);
    if ($a->{x} == $b->{x}) {
        $t->{table}->[$a->{y}][($a->{x} + 1) % 5] .
        $t->{table}->[$b->{y}][($b->{x} + 1) % 5]
    } elsif ($a->{y} == $b->{y}) {
        $t->{table}->[($a->{y} + 1) % 5][$a->{x}] .
        $t->{table}->[($b->{y} + 1) % 5][$b->{x}]
    } else {
        $t->{table}->[$a->{y}][$b->{x}] .
        $t->{table}->[$b->{y}][$a->{x}]
    }
}

sub decode_pair {
    my ($t, $pair) = @_;
    my ($a, $b) = (map { $t->{coords}->{$_} } @$pair);
    if ($a->{x} == $b->{x}) {
        $t->{table}->[$a->{y}][($a->{x} - 1) % 5] .
        $t->{table}->[$b->{y}][($b->{x} - 1) % 5]
    } elsif ($a->{y} == $b->{y}) {
        $t->{table}->[($a->{y} - 1) % 5][$a->{x}] .
        $t->{table}->[($b->{y} - 1) % 5][$b->{x}]
    } else {
        $t->{table}->[$a->{y}][$b->{x}] .
        $t->{table}->[$b->{y}][$a->{x}]
    }
}

sub encode {
    my ($key, $text) = @_;
    my $table = table($key);
    join '', map { encode_pair($table, $_) } split_to_pairs($text)
}

sub decode {
    my ($key, $text) = @_;
    my $table = table($key);
    join '', map { decode_pair($table, $_) } split_to_pairs($text)
}

sub print_as_pairs {
    my $i = 0;
    for (split_to_pairs($_[0])) {
        print join '', @$_;
        print ' ';
        print "\n" unless ++$i % 9;
    }
    print "\n"
}

my $key = 'OIL';
my $res = encode($key, 'BYHAPY');
print_as_pairs($res);
print_as_pairs(decode($key, $res));


\end{lstlisting}

Результат работы программы: \\
AZ NO NZ \\
BY HA PY \\

Программа так же тестировалась на примере из книги 
Саломаа А. "КРИПТОГРАФИЯ С ОТКРЫТЫМ КЛЮЧОМ" стр. 39, 43.

ключ: OIL

шифротекст:

QN FS LK CM LT HC SM MC VK \\
IH HA XR QM BQ IE QN AK RD \\
PS TU CB NX MC IF NX MC IT \\
YF SD EF IF QN LQ FL YD SB \\
QN AK EU MC TI IE QN MS IQ \\
KA PF IL BM WD DF RE IV KA \\
MC IT QN FX MB FT FT DX AK \\
HC SM YF WE BA AB QE IV OI \\
XT IT FM AQ AK QN MX ZU DS \\
OI XI QN FY RX NV OR RB RA \\
MC MB NX XM AE OW FT LR NC \\
IQ QN FM ML SN AH QN QL TW \\
FL ST LT PI QI QN DS VK AR \\
FS AQ TI DF SM AK FO XM VA \\
RZ FT SN GS UD FM SA WA LN \\
MF IT QN FG LN BQ QE AR VA \\
DT FT QA AB FY IT MX DK FM \\
DF QN FX NO XC TF SM FK OY \\
CM QM BA LH \\

исходный текст:

TH ET IM EH AS CO ME HE WH \\
OK NO WS SH OU LD TH IN KI \\
MU ST GO MY HE AD MY HE AR \\
TA RE DE AD TH OS EA WF UL \\
TH IN GS HE RA LD TH EM OR \\
NI NG OI LP RI CE SD OW NI \\
HE AR TH EY PL AN AN EW IN \\
CO ME TA XD AL LA SC OW BO \\
YS AR EN OT IN TH ES UP ER \\
BO WL TH AT SW HY IQ UI TI \\
HE LP MY SE LF IV AN IS HF \\
OR TH EN EX TM ON TH SO RY \\
EA RS AS KB RO TH ER WH IT \\
ET OT RA CE ME IN CA SE YO \\
UW AN TM EU RG EN TL YI AM \\
NE AR TH EF AM OU SC IT YO \\
FR AN TO LA AT AR ES ID EN \\
CE TH EY HA VE NA ME DN AV \\
EH SH AL OM \\

Для криптоанализа необходимо иметь статистику о частоте вхождения пар
букв(диграмм) в предложениях языка, на котором написан текст.

\end{document}
